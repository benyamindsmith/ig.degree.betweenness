\documentclass[10pt,a4paper,onecolumn]{article}
\usepackage{marginnote}
\usepackage{graphicx}
\usepackage{xcolor}
\usepackage{authblk,etoolbox}
\usepackage{titlesec}
\usepackage{calc}
\usepackage{tikz}
\usepackage{hyperref}
\hypersetup{colorlinks,breaklinks,
            urlcolor=[rgb]{0.0, 0.5, 1.0},
            linkcolor=[rgb]{0.0, 0.5, 1.0}}
\usepackage{caption}
\usepackage{tcolorbox}
\usepackage{amssymb,amsmath}
\usepackage{ifxetex,ifluatex}
\usepackage{seqsplit}
\usepackage{fixltx2e} % provides \textsubscript
\usepackage[
  backend=biber,
%  style=alphabetic,
%  citestyle=numeric
]{biblatex}
\bibliography{paper.bib}



% --- Page layout -------------------------------------------------------------
\usepackage[top=3.5cm, bottom=3cm, right=1.5cm, left=1.0cm,
            headheight=2.2cm, reversemp, includemp, marginparwidth=4.5cm]{geometry}

% --- Default font ------------------------------------------------------------
% \renewcommand\familydefault{\sfdefault}

% --- Style -------------------------------------------------------------------
\renewcommand{\bibfont}{\small \sffamily}
\renewcommand{\captionfont}{\small\sffamily}
\renewcommand{\captionlabelfont}{\bfseries}

% --- Section/SubSection/SubSubSection ----------------------------------------
\titleformat{\section}
  {\normalfont\sffamily\Large\bfseries}
  {}{0pt}{}
\titleformat{\subsection}
  {\normalfont\sffamily\large\bfseries}
  {}{0pt}{}
\titleformat{\subsubsection}
  {\normalfont\sffamily\bfseries}
  {}{0pt}{}
\titleformat*{\paragraph}
  {\sffamily\normalsize}


% --- Header / Footer ---------------------------------------------------------
\usepackage{fancyhdr}
\pagestyle{fancy}
\fancyhf{}
%\renewcommand{\headrulewidth}{0.50pt}
\renewcommand{\headrulewidth}{0pt}
\fancyhead[L]{\hspace{-0.75cm}\includegraphics[width=5.5cm]{C:/Users/ben29/AppData/Local/R/win-library/4.2/rticles/rmarkdown/templates/joss/resources/JOSS-logo.png}}
\fancyhead[C]{}
\fancyhead[R]{}
\renewcommand{\footrulewidth}{0.25pt}

\fancyfoot[L]{\footnotesize{\sffamily , (). ig.degree.betweenness: A
Community Detection Algorithm Leveraging Degree
Centrality. \textit{Journal of Open Source Software}, (), . \href{https://doi.org/}{https://doi.org/}}}


\fancyfoot[R]{\sffamily \thepage}
\makeatletter
\let\ps@plain\ps@fancy
\fancyheadoffset[L]{4.5cm}
\fancyfootoffset[L]{4.5cm}

% --- Macros ---------

\definecolor{linky}{rgb}{0.0, 0.5, 1.0}

\newtcolorbox{repobox}
   {colback=red, colframe=red!75!black,
     boxrule=0.5pt, arc=2pt, left=6pt, right=6pt, top=3pt, bottom=3pt}

\newcommand{\ExternalLink}{%
   \tikz[x=1.2ex, y=1.2ex, baseline=-0.05ex]{%
       \begin{scope}[x=1ex, y=1ex]
           \clip (-0.1,-0.1)
               --++ (-0, 1.2)
               --++ (0.6, 0)
               --++ (0, -0.6)
               --++ (0.6, 0)
               --++ (0, -1);
           \path[draw,
               line width = 0.5,
               rounded corners=0.5]
               (0,0) rectangle (1,1);
       \end{scope}
       \path[draw, line width = 0.5] (0.5, 0.5)
           -- (1, 1);
       \path[draw, line width = 0.5] (0.6, 1)
           -- (1, 1) -- (1, 0.6);
       }
   }

% --- Title / Authors ---------------------------------------------------------
% patch \maketitle so that it doesn't center
\patchcmd{\@maketitle}{center}{flushleft}{}{}
\patchcmd{\@maketitle}{center}{flushleft}{}{}
% patch \maketitle so that the font size for the title is normal
\patchcmd{\@maketitle}{\LARGE}{\LARGE\sffamily}{}{}
% patch the patch by authblk so that the author block is flush left
\def\maketitle{{%
  \renewenvironment{tabular}[2][]
    {\begin{flushleft}}
    {\end{flushleft}}
  \AB@maketitle}}
\makeatletter
\renewcommand\AB@affilsepx{ \protect\Affilfont}
%\renewcommand\AB@affilnote[1]{{\bfseries #1}\hspace{2pt}}
\renewcommand\AB@affilnote[1]{{\bfseries #1}\hspace{3pt}}
\makeatother
\renewcommand\Authfont{\sffamily\bfseries}
\renewcommand\Affilfont{\sffamily\small\mdseries}
\setlength{\affilsep}{1em}


\ifnum 0\ifxetex 1\fi\ifluatex 1\fi=0 % if pdftex
  \usepackage[T1]{fontenc}
  \usepackage[utf8]{inputenc}

\else % if luatex or xelatex
  \ifxetex
    \usepackage{mathspec}
  \else
    \usepackage{fontspec}
  \fi
  \defaultfontfeatures{Ligatures=TeX,Scale=MatchLowercase}

\fi
% use upquote if available, for straight quotes in verbatim environments
\IfFileExists{upquote.sty}{\usepackage{upquote}}{}
% use microtype if available
\IfFileExists{microtype.sty}{%
\usepackage{microtype}
\UseMicrotypeSet[protrusion]{basicmath} % disable protrusion for tt fonts
}{}

\usepackage{hyperref}
\hypersetup{unicode=true,
            pdftitle={ig.degree.betweenness: A Community Detection Algorithm Leveraging Degree Centrality},
            pdfauthor={Benjamin Smith},
            pdfborder={0 0 0},
            breaklinks=true}
\urlstyle{same}  % don't use monospace font for urls
\usepackage{graphicx,grffile}
\makeatletter
\def\maxwidth{\ifdim\Gin@nat@width>\linewidth\linewidth\else\Gin@nat@width\fi}
\def\maxheight{\ifdim\Gin@nat@height>\textheight\textheight\else\Gin@nat@height\fi}
\makeatother
% Scale images if necessary, so that they will not overflow the page
% margins by default, and it is still possible to overwrite the defaults
% using explicit options in \includegraphics[width, height, ...]{}
\setkeys{Gin}{width=\maxwidth,height=\maxheight,keepaspectratio}
\IfFileExists{parskip.sty}{%
\usepackage{parskip}
}{% else
\setlength{\parindent}{0pt}
\setlength{\parskip}{6pt plus 2pt minus 1pt}
}
\setlength{\emergencystretch}{3em}  % prevent overfull lines
\setcounter{secnumdepth}{0}
% Redefines (sub)paragraphs to behave more like sections
\ifx\paragraph\undefined\else
\let\oldparagraph\paragraph
\renewcommand{\paragraph}[1]{\oldparagraph{#1}\mbox{}}
\fi
\ifx\subparagraph\undefined\else
\let\oldsubparagraph\subparagraph
\renewcommand{\subparagraph}[1]{\oldsubparagraph{#1}\mbox{}}
\fi

% Pandoc syntax highlighting
\usepackage{color}
\usepackage{fancyvrb}
\newcommand{\VerbBar}{|}
\newcommand{\VERB}{\Verb[commandchars=\\\{\}]}
\DefineVerbatimEnvironment{Highlighting}{Verbatim}{commandchars=\\\{\}}
% Add ',fontsize=\small' for more characters per line
\usepackage{framed}
\definecolor{shadecolor}{RGB}{248,248,248}
\newenvironment{Shaded}{\begin{snugshade}}{\end{snugshade}}
\newcommand{\AlertTok}[1]{\textcolor[rgb]{0.94,0.16,0.16}{#1}}
\newcommand{\AnnotationTok}[1]{\textcolor[rgb]{0.56,0.35,0.01}{\textbf{\textit{#1}}}}
\newcommand{\AttributeTok}[1]{\textcolor[rgb]{0.13,0.29,0.53}{#1}}
\newcommand{\BaseNTok}[1]{\textcolor[rgb]{0.00,0.00,0.81}{#1}}
\newcommand{\BuiltInTok}[1]{#1}
\newcommand{\CharTok}[1]{\textcolor[rgb]{0.31,0.60,0.02}{#1}}
\newcommand{\CommentTok}[1]{\textcolor[rgb]{0.56,0.35,0.01}{\textit{#1}}}
\newcommand{\CommentVarTok}[1]{\textcolor[rgb]{0.56,0.35,0.01}{\textbf{\textit{#1}}}}
\newcommand{\ConstantTok}[1]{\textcolor[rgb]{0.56,0.35,0.01}{#1}}
\newcommand{\ControlFlowTok}[1]{\textcolor[rgb]{0.13,0.29,0.53}{\textbf{#1}}}
\newcommand{\DataTypeTok}[1]{\textcolor[rgb]{0.13,0.29,0.53}{#1}}
\newcommand{\DecValTok}[1]{\textcolor[rgb]{0.00,0.00,0.81}{#1}}
\newcommand{\DocumentationTok}[1]{\textcolor[rgb]{0.56,0.35,0.01}{\textbf{\textit{#1}}}}
\newcommand{\ErrorTok}[1]{\textcolor[rgb]{0.64,0.00,0.00}{\textbf{#1}}}
\newcommand{\ExtensionTok}[1]{#1}
\newcommand{\FloatTok}[1]{\textcolor[rgb]{0.00,0.00,0.81}{#1}}
\newcommand{\FunctionTok}[1]{\textcolor[rgb]{0.13,0.29,0.53}{\textbf{#1}}}
\newcommand{\ImportTok}[1]{#1}
\newcommand{\InformationTok}[1]{\textcolor[rgb]{0.56,0.35,0.01}{\textbf{\textit{#1}}}}
\newcommand{\KeywordTok}[1]{\textcolor[rgb]{0.13,0.29,0.53}{\textbf{#1}}}
\newcommand{\NormalTok}[1]{#1}
\newcommand{\OperatorTok}[1]{\textcolor[rgb]{0.81,0.36,0.00}{\textbf{#1}}}
\newcommand{\OtherTok}[1]{\textcolor[rgb]{0.56,0.35,0.01}{#1}}
\newcommand{\PreprocessorTok}[1]{\textcolor[rgb]{0.56,0.35,0.01}{\textit{#1}}}
\newcommand{\RegionMarkerTok}[1]{#1}
\newcommand{\SpecialCharTok}[1]{\textcolor[rgb]{0.81,0.36,0.00}{\textbf{#1}}}
\newcommand{\SpecialStringTok}[1]{\textcolor[rgb]{0.31,0.60,0.02}{#1}}
\newcommand{\StringTok}[1]{\textcolor[rgb]{0.31,0.60,0.02}{#1}}
\newcommand{\VariableTok}[1]{\textcolor[rgb]{0.00,0.00,0.00}{#1}}
\newcommand{\VerbatimStringTok}[1]{\textcolor[rgb]{0.31,0.60,0.02}{#1}}
\newcommand{\WarningTok}[1]{\textcolor[rgb]{0.56,0.35,0.01}{\textbf{\textit{#1}}}}

% tightlist command for lists without linebreak
\providecommand{\tightlist}{%
  \setlength{\itemsep}{0pt}\setlength{\parskip}{0pt}}


% Pandoc citation processing
%From Pandoc 3.1.8
% definitions for citeproc citations
\NewDocumentCommand\citeproctext{}{}
\NewDocumentCommand\citeproc{mm}{%
  \begingroup\def\citeproctext{#2}\cite{#1}\endgroup}
\makeatletter
 % allow citations to break across lines
 \let\@cite@ofmt\@firstofone
 % avoid brackets around text for \cite:
 \def\@biblabel#1{}
 \def\@cite#1#2{{#1\if@tempswa , #2\fi}}
\makeatother
\newlength{\cslhangindent}
\setlength{\cslhangindent}{1.5em}
\newlength{\csllabelwidth}
\setlength{\csllabelwidth}{3em}
\newenvironment{CSLReferences}[2] % #1 hanging-indent, #2 entry-spacing
 {\begin{list}{}{%
  \setlength{\itemindent}{0pt}
  \setlength{\leftmargin}{0pt}
  \setlength{\parsep}{0pt}
  % turn on hanging indent if param 1 is 1
  \ifodd #1
   \setlength{\leftmargin}{\cslhangindent}
   \setlength{\itemindent}{-1\cslhangindent}
  \fi
  % set entry spacing
  \setlength{\itemsep}{#2\baselineskip}}}
 {\end{list}}
\usepackage{calc}
\newcommand{\CSLBlock}[1]{#1\hfill\break}
\newcommand{\CSLLeftMargin}[1]{\parbox[t]{\csllabelwidth}{#1}}
\newcommand{\CSLRightInline}[1]{\parbox[t]{\linewidth - \csllabelwidth}{#1}\break}
\newcommand{\CSLIndent}[1]{\hspace{\cslhangindent}#1}



\title{ig.degree.betweenness: A Community Detection Algorithm Leveraging
Degree Centrality}

        \author[1]{Benjamin Smith}
          \author{Tyler Pittman}
          \author[12]{Wei Xu}
    
      \affil[1]{University of Toronto}
      \affil[2]{Princess Margaret Cancer Centre: Toronto, Ontario, CA}
  \date{\vspace{-5ex}}

\begin{document}
\maketitle

\marginpar{
  %\hrule
  \sffamily\small

  {\bfseries DOI:} \href{https://doi.org/}{\color{linky}{}}

  \vspace{2mm}

  {\bfseries Software}
  \begin{itemize}
    \setlength\itemsep{0em}
    \item \href{}{\color{linky}{Review}} \ExternalLink
    \item \href{}{\color{linky}{Repository}} \ExternalLink
    \item \href{}{\color{linky}{Archive}} \ExternalLink
  \end{itemize}

  \vspace{2mm}

  {\bfseries Submitted:} \\
  {\bfseries Published:} 

  \vspace{2mm}
  {\bfseries License}\\
  Authors of papers retain copyright and release the work under a Creative Commons Attribution 4.0 International License (\href{http://creativecommons.org/licenses/by/4.0/}{\color{linky}{CC-BY}}).
}

\section{Summary}\label{summary}

\{ig.degree.betweenness\} is an R (R Core Team 2022) package which
implements the ``Smith-Pittman'' community detection algorithm (Smith,
Pittman, and Xu 2024) and is directly compatible with networks and
sociograms constructed and loaded with \texttt{igraph} package (Csárdi
et al. 2024) by Csardi and Nepusz (Csardi and Nepusz 2006).
\{ig.degree.betweenness\} also offers utility functions to which enable
neater plotting of densely connected networks and preparation of
unlabeled graphs for algorithm implementation.

\section{Statement of Need}\label{statement-of-need}

The \texttt{igraph} package offers a suite of community detection
algorithms, including Girvan-Newman (Girvan and Newman 2002) and Louvain
(Blondel et al. 2008). In densely connected complex networks it has been
noted by (Smith, Pittman, and Xu 2024) that

\section{Minimal Example}\label{minimal-example}

\subsection{Zachary's Karate Club
Network}\label{zacharys-karate-club-network}

The dataset commonly referred to as ``Zachary's karate club network'' by
Zachary (1997) is a social network between members of a university club
led by president John A. and karate instructor Mr.~Hi (pseudonyms). At
the beginning of the study there was an initial conflict between the
club president, John A., and Mr.~Hi over the price of karate lessons. As
time passed, the entire club became divided over this issue. After a
series of increasingly sharp factional confrontations over the price of
lessons, the officers of the club, led by John A., fired Mr.~Hi. The
supporters of Mr.~Hi retaliated by resigning and forming a new
organization headed by Mr.~Hi. Figure 1 shows the karate club network
where the nodes signify individuals in the club and the edges signifies
the existence of a relationship between two members. The node color
indicates which group the members associated with post-split.

Since the division of the club and its members is known, this social
network is a classic example dataset used and studied. In the context of
community detection, the object of interest is seeing if the split could
be identified based on the relationships between members. When applied
in an unsupervised setting, the Girvan-Newman and Louvain algorthims
identify communities of nodes which optimize modularity according to
their approaches. However, the communities identified do not appear to
identify a possible division in the group which is contextually
informative or interpretative. The Smith-Pittman algorithm identifies 3
communities which could can be understood as individuals who would
certainly associate with John A. or Mr.~Hi and an uncertain group.
Figure 2 shows the comparison between the three algorithms.

\begin{Shaded}
\begin{Highlighting}[]
\CommentTok{\# Install packages}
\CommentTok{\# install.packages(c("igraph","igraphdata", "ig.degree.betweenness"))}

\FunctionTok{set.seed}\NormalTok{(}\DecValTok{5250}\NormalTok{) }\CommentTok{\#Setting seed to visual reproducibility}
\FunctionTok{library}\NormalTok{(igraph)}
\FunctionTok{library}\NormalTok{(igraphdata)}
\FunctionTok{library}\NormalTok{(ig.degree.betweenness)}

\FunctionTok{data}\NormalTok{(}\StringTok{"karate"}\NormalTok{)}

\FunctionTok{par}\NormalTok{(}\AttributeTok{mar=}\FunctionTok{c}\NormalTok{(}\DecValTok{0}\NormalTok{,}\DecValTok{0}\NormalTok{,}\DecValTok{0}\NormalTok{,}\DecValTok{0}\NormalTok{)}\SpecialCharTok{+}\NormalTok{.}\DecValTok{1}\NormalTok{)}
\FunctionTok{plot}\NormalTok{(karate)}
\end{Highlighting}
\end{Shaded}

\begin{figure}
\centering
\pandocbounded{\includegraphics[keepaspectratio]{./images/karate_network.png}}
\caption{The Zachary karate club network with the true split between
members defined by node colors. John A. and Mr.~Hi are denoted by `J'
and `H', with other members being listed as numbers}
\end{figure}

\begin{Shaded}
\begin{Highlighting}[]
\NormalTok{gn\_karate }\OtherTok{\textless{}{-}}\NormalTok{ karate }\SpecialCharTok{|\textgreater{}}
\NormalTok{  igraph}\SpecialCharTok{::}\FunctionTok{cluster\_edge\_betweenness}\NormalTok{()}

\NormalTok{louvain\_karate }\OtherTok{\textless{}{-}}\NormalTok{ karate }\SpecialCharTok{|\textgreater{}}
\NormalTok{  igraph}\SpecialCharTok{::}\FunctionTok{cluster\_louvain}\NormalTok{()}

\NormalTok{sp\_karate }\OtherTok{\textless{}{-}}\NormalTok{ karate }\SpecialCharTok{|\textgreater{}}
\NormalTok{  ig.degree.betweenness}\SpecialCharTok{::}\FunctionTok{cluster\_degree\_betweenness}\NormalTok{()}

\FunctionTok{par}\NormalTok{(}\AttributeTok{mfrow=} \FunctionTok{c}\NormalTok{(}\DecValTok{1}\NormalTok{,}\DecValTok{3}\NormalTok{),}\AttributeTok{mar=}\FunctionTok{c}\NormalTok{(}\DecValTok{0}\NormalTok{,}\DecValTok{0}\NormalTok{,}\DecValTok{0}\NormalTok{,}\DecValTok{0}\NormalTok{)}\SpecialCharTok{+}\DecValTok{1}\NormalTok{)}

\FunctionTok{plot}\NormalTok{(}
\NormalTok{  gn\_karate,}
\NormalTok{  karate,}
  \AttributeTok{main =} \StringTok{"(a)"}
\NormalTok{  )}

\FunctionTok{plot}\NormalTok{(}
\NormalTok{  louvain\_karate,}
\NormalTok{  karate,}
  \AttributeTok{main =} \StringTok{"(b)"}
\NormalTok{)}

\FunctionTok{plot}\NormalTok{(}
\NormalTok{  sp\_karate,}
\NormalTok{  karate,}
  \AttributeTok{main =} \StringTok{"(c)"}
\NormalTok{)}
\end{Highlighting}
\end{Shaded}

\begin{figure}
\centering
\pandocbounded{\includegraphics[keepaspectratio]{./images/algorithm_comparison_karate.png}}
\caption{Unsupervised Community Detection by (a) Girvan-Newman, (b)
Louvain and (c) Smith-Pittman for the karate network.}
\end{figure}

\section{Acknowledgements}\label{acknowledgements}

The author expresses gratitude towards Tyler Pittman and Dr.~Wei Xu for
their invaluable feedback in developing the methodology in this paper.

\section*{References}\label{references}
\addcontentsline{toc}{section}{References}

\phantomsection\label{refs}
\begin{CSLReferences}{1}{0}
\bibitem[\citeproctext]{ref-louvain_paper}
Blondel, Vincent D, Jean-Loup Guillaume, Renaud Lambiotte, and Etienne
Lefebvre. 2008. {``Fast Unfolding of Communities in Large Networks.''}
\emph{Journal of Statistical Mechanics Theory and Experiment} 2008 (10):
P10008. \url{https://doi.org/10.1088/1742-5468/2008/10/p10008}.

\bibitem[\citeproctext]{ref-igraph_article}
Csardi, Gabor, and Tamas Nepusz. 2006. {``The Igraph Software Package
for Complex Network Research.''} \emph{InterJournal} Complex Systems:
1695. \url{https://igraph.org}.

\bibitem[\citeproctext]{ref-igraph_software}
Csárdi, Gábor, Tamás Nepusz, Vincent Traag, Szabolcs Horvát, Fabio
Zanini, Daniel Noom, and Kirill Müller. 2024. \emph{{igraph}: Network
Analysis and Visualization in r}.
\url{https://doi.org/10.5281/zenodo.7682609}.

\bibitem[\citeproctext]{ref-Girvan_Newman_2002}
Girvan, M., and M. E. J. Newman. 2002. {``Community Structure in Social
and Biological Networks.''} \emph{Proceedings of the National Academy of
Sciences} 99 (12): 7821--26.
\url{https://doi.org/10.1073/pnas.122653799}.

\bibitem[\citeproctext]{ref-base2022}
R Core Team. 2022. \emph{R: A Language and Environment for Statistical
Computing}. Vienna, Austria: R Foundation for Statistical Computing.
\url{https://www.R-project.org/}.

\bibitem[\citeproctext]{ref-sp_abstract}
Smith, Benjamin, Tyler Pittman, and Wei Xu. 2024. {``Centrality in
Collaboration: Community Detection for Oncology Researchers.''}
\emph{University of Toronto Journal of Public Health} TODO.
\url{https://arxiv.org}.

\end{CSLReferences}

\end{document}
